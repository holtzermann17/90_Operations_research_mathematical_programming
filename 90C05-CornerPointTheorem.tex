\documentclass[12pt]{article}
\usepackage{pmmeta}
\pmcanonicalname{CornerPointTheorem}
\pmcreated{2013-03-22 15:39:16}
\pmmodified{2013-03-22 15:39:16}
\pmowner{CWoo}{3771}
\pmmodifier{CWoo}{3771}
\pmtitle{corner point theorem}
\pmrecord{12}{37587}
\pmprivacy{1}
\pmauthor{CWoo}{3771}
\pmtype{Theorem}
\pmcomment{trigger rebuild}
\pmclassification{msc}{90C05}
\pmsynonym{fundamental theorem of linear programming}{CornerPointTheorem}
\pmdefines{corner point}

\endmetadata

\usepackage{amssymb,amscd}
\usepackage{amsmath}
\usepackage{amsfonts}

% used for TeXing text within eps files
%\usepackage{psfrag}
% need this for including graphics (\includegraphics)
%\usepackage{graphicx}
% for neatly defining theorems and propositions
%\usepackage{amsthm}
% making logically defined graphics
%%%\usepackage{xypic}

% define commands here
\begin{document}
Let $P$ be a primal linear programming problem with $f$ the objective function and polyhedron $R$ as its feasible region.  A \emph{corner point}, or extreme point of $P$ is a \PMlinkescapetext{vertex} of $R$.

\textbf{Corner Point Theorem.}  If $P$ has an optimal solution $a<\infty$, then there is a corner point $p$ of $P$ such that $f(p)=a$.  If another corner point $q$ also satisfies $f(x)=a$, then $f(\overline{pq})=\lbrace a\rbrace$.  If $r$ is a third corner point such that $f(r)=a$, then $f(\triangle pqr)=\lbrace a\rbrace$.

Note that the line segment and triangle mentioned above are necessarily subsets of $R$.

A cousin of the above theorem is the following:

\textbf{Theorem.}  If $P$ has an optimal solution $a<\infty$ occurring at an interior point of an edge $E$ (or a face $F$) on the boundary of the feasible region $R$, then $f(E)=\lbrace a\rbrace$ (or $f(F)=\lbrace a\rbrace$).  In fact, if the solution occurs at an interior point of $R$, then the solution is satisfied by all points of $R$: $f(R)=\lbrace a\rbrace$.  In other words, the objective function $f$ is constant on $R$.

On way to visualize the above theorems is to simplify them into the case when the objective function is a line $f(x)=mx+b$ on the ``$x-y$ plane'' and the feasible region is a line segment $I=[x_0,x_1]$ on the $x$-axis.  It is easy to see now that the maximum (or minimum) of $f$ occurs at either $x_0$ or $x_1$.  If the optimal value occurs either at both end points, or at an interior point $x_2\in(x_0,x_1)$, then $f$ is a horizontal line segment on $I$.

An application of the above theorems can be demonstrated by the following example:  If the feasible region $R$ is a unit square and if corner points $(0,0),(1,1)$ satisfy the optimal solution of $P$, then all points on $\lbrace (x,y)\mid x=y\rbrace \cap R$ satisfy the solution.  In particular, $(\frac{1}{2},\frac{1}{2})$, an interior point of $R$, satisfies the solution.  As a result, all points of $R$ satisfy the solution.
%%%%%
%%%%%
\end{document}
