\documentclass[12pt]{article}
\usepackage{pmmeta}
\pmcanonicalname{SimplexAlgorithm}
\pmcreated{2013-03-22 13:35:21}
\pmmodified{2013-03-22 13:35:21}
\pmowner{Mathprof}{13753}
\pmmodifier{Mathprof}{13753}
\pmtitle{simplex algorithm}
\pmrecord{22}{34212}
\pmprivacy{1}
\pmauthor{Mathprof}{13753}
\pmtype{Algorithm}
\pmcomment{trigger rebuild}
\pmclassification{msc}{90C05}
\pmsynonym{simplex method}{SimplexAlgorithm}
%\pmkeywords{linear programming}

\endmetadata

% this is the default PlanetMath preamble.  as your knowledge
% of TeX increases, you will probably want to edit this, but
% it should be fine as is for beginners.

% almost certainly you want these
\usepackage{amssymb}
\usepackage{amsmath}
\usepackage{amsfonts}

% used for TeXing text within eps files
%\usepackage{psfrag}
% need this for including graphics (\includegraphics)
%\usepackage{graphicx}
% for neatly defining theorems and propositions
%\usepackage{amsthm}
% making logically defined graphics
%%%\usepackage{xypic}

% there are many more packages, add them here as you need them

% define commands here
\newcommand{\R}{\ensuremath{\mathbb{R}}}
\begin{document}
\PMlinkescapeword{certificate}
\PMlinkescapeword{vertex}
\PMlinkescapeword{vertices}
The simplex algorithm is  used as part of the simplex
method (due to George B. Dantzig) to solve linear programming problems.
The algorithm  is applied to a linear programming problem that
is in canonical form. 

A \emph{canonical system} of equations has an ordered subset of variables
(called the \emph{basis}) such that for each $i$, the $i^{th}$ basic 
variable has a unit coefficient in the $i^{th}$ equation and zero coefficient
in the other equations.

As an example $x_1, \ldots , x_r$ are basic variables in the following
system of $r$ equations:

\begin{eqnarray*}
x_1 \quad \quad &+& a_{1,r+1}x_{r+1} + \cdots + a_{1,n}x_n = b_1 \\
x_2 \quad  &+& a_{2,r+1}x_{r+1}+ \cdots + a_{2,n}x_n = b_2 \\
\ldots \\
x_r   &+& a_{r,r+1}x_{r+1} + \cdots + a_{r,n}x_n = b_r \\
\end{eqnarray*}



The simplex algorithm is used as one phase of the simplex method.

Suppose that we have a canonical system with basic variables 
$x_1, \ldots, x_m, -z$ and we seek to find nonnegative $x_i$ $i=1, \ldots, n$
such that $z$ is minimal. That is, we have

\begin{eqnarray*}
x_i + \sum_{j=m+1}^n a_{ij} x_j &=& b_i \quad i=1, \ldots, m \\
-z + \sum_{j=m+1}^n c_j x_j &=& -z_o \\
\end{eqnarray*}
where $a_{ij}, b_j , c_j , z_o$ are constants, 
and $b_j \geq 0, \quad j=1,\ldots, m$.

Notice that if we set $x_{m+1} = 0, \ldots, x_n = 0$ we will have a feasible
solution with $z=z_o.$. Hence, any optimal solution will have
$z \leq z_o$. 
The algorithm can now be described as follows:

Step 1. Set $N = \{m+1, \ldots , n\}$ and $B=\{1, \ldots, m\}$.
Put $c_j = 0$ for $j\in B$.

Step 2. If there an index $j \in N$ such that $c_j < 0$ then
choose $s \in N$ such that 
$$
c_s = \operatorname{min}_{j \in N} c_j 
$$
else
stop. The solution is given by $x_i = 0$ for $i \in N$ and 
$x_i = b_i$ for $i \in B$, $z=z_o$.

Step 3. If $a_{is} \leq 0$ for all $i$ then stop. The value of $z$ 
has no lower bound.
Else, let $\frac{b_r}{a_{rs}} = \operatorname{min}_{a_{is}>0} \frac{b_i}{a_{is}}$.
If there is more than one choice for $r$ it does not matter which one
is chosen \emph{unless} $b_i = 0$. This is the so-called degenerate case.
In this case, one can choose uniformly at random from among those
$i$ for which $b_i = 0$.

Step 4. (Pivot on $a_{rs}$). Multiply the $r^{th}$ equation by $\frac{1}{a_{rs}}$ and for each $i=1, \ldots, m$, $i \ne r$
replace equation $i$ by the sum of equation $i$ and the (replaced) equation
$r$ multiplied by $-a_{is}$. Replace the equation for $z$ by
the  sum of the equation for $z$ and the (replaced) equation 
$r$ multiplied by $-c_s$. Note: The replacement operations of course
change the coefficients $a_{ij}$ and $c_j$. As the algorithm proceeds
it is of course necessary to use the changed coefficients. 

Step 5. (Update $B$ and $N$) Put $s$ into $B$ and $r$ into $N$
and remove $s$ from $N$ and $r$ from $B$. Go to step 2.

There are examples where the algorithm does not terminate in a finite
number of steps; but if there is non-degeneracy at each iteration,
the algorithm will terminate in a finite number of steps.
%%%%%
%%%%%
\end{document}
